\chapter*{General Conclusion and Future Work}
\addcontentsline{toc}{chapter}{General Conclusion and Future Work}
\markboth{General Conclusion and Future Work}{General Conclusion and Future Work}


This thesis presented MSFusionXAI, an explainable multimodal framework for automated Multiple Sclerosis diagnosis combining MRI imaging with clinical biomarkers.  Validated on 92 subjects from CHU Sahloul University Hospital, the system achieved 93.7\% classification accuracy and 0.823 segmentation Dice coefficient, demonstrating practical viability for single-center clinical deployment.

The framework's key contributions span four areas.  First, a comprehensive MRI preprocessing pipeline incorporating bias correction, normalization, and harmonization proved essential, with ablation studies showing intensity normalization alone contributed 7.8\% to classification accuracy. Second, the novel adaptive attention mechanism dynamically weighted MRI and clinical features per patient, outperforming MRI-only (72. 6\%) and clinical-only (89.4\%) baselines while reducing cross-fold variability by 61\%, demonstrating robustness critical for clinical trust. Third, U-Net with ResNet34 encoder achieved mean Dice of 0. 823 on held-out test patients, competitive with state-of-the-art methods despite training on only 46 MS patients versus 200-1000+ cases in prior works. Fourth, multi-level explainability through attention weight analysis, lesion correlation studies, and BioBERT-generated clinical reports validated by neurologists provided comprehensive transparency addressing regulatory requirements.

The system processes one patient in 9 seconds on standard GPU hardware, enabling real-time clinical integration. The adaptive attention mechanism mirrors neurologist reasoning by prioritizing imaging for high lesion burden cases while emphasizing clinical biomarkers when imaging is subtle, enhancing clinical acceptance. However, the single-center retrospective design limits generalization to broader populations and scanner protocols. Manual annotations by one neurologist introduce subjective variability, the 2D slice-based encoder may miss 3D lesion patterns, and limited clinical features omit disease duration and treatment history. Despite these constraints, MSFusionXAI demonstrated that explainable multimodal AI is viable for MS diagnosis support in resource-limited settings. 

\section*{Future Work}

Several realistic directions could enhance the framework's impact. Multi-center validation through collaboration with MS consortia (MAGNIMS, NAIMS) would assess generalization across scanners, populations, and disease subtypes, potentially using federated learning to preserve patient privacy. Extending to longitudinal modeling with sequential MRI and clinical data could forecast disability progression and treatment response, enabling personalized therapy selection.  Replacing 2D encoding with 3D convolutional networks would capture volumetric lesion context, though requiring more computational resources.  Incorporating multiple MRI sequences (T1-weighted, T2-weighted, diffusion-weighted imaging) through multi-stream architectures could improve lesion characterization and differentiate MS from mimicking diseases. 

Self-supervised pre-training on large unlabeled MRI datasets using contrastive learning or masked image modeling could improve feature representations, reducing reliance on expensive manual annotations.  Enhanced explainability through Grad-CAM or SHAP values would provide pixel-level explanations, while fine-tuning BioBERT on MS-specific literature would improve report terminology and clinical relevance. Most critically, prospective clinical trials with real-world deployment at CHU Sahloul as a PACS-integrated plugin could quantify clinical impact on diagnostic accuracy, time-to-diagnosis, and patient outcomes through randomized controlled comparison with standard workflows.  Pursuing regulatory compliance (FDA 510(k), CE marking) would require rigorous validation protocols, bias audits across demographic groups, and establishing clinical decision support guidelines defining appropriate use cases and human oversight requirements. 

\section*{Closing Remarks}

This work demonstrated that explainable multimodal AI can deliver clinically acceptable MS diagnostic support even with limited single-center data. By combining adaptive fusion, conditional processing, and natural language explanation, MSFusionXAI bridges the gap between AI capabilities and clinical requirements for transparency.  The framework provides a practical blueprint for developing trustworthy medical AI in resource-constrained settings, emphasizing that sophisticated diagnostic tools need not require massive datasets or specialized infrastructure.  Future research building on these foundations can advance AI-assisted precision medicine that improves diagnostic accuracy, reduces time-to-diagnosis, and ultimately enhances patient outcomes through earlier, more confident MS diagnosis and personalized treatment strategies. 